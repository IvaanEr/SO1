
\documentclass[a4paper]{article}
\usepackage[utf8]{inputenc}
\usepackage{amsmath,amssymb,latexsym}
\usepackage{syntax}
\usepackage[margin=1.5cm]{geometry}
\usepackage{amssymb}
\usepackage{tipa}
\usepackage{listings}
\usepackage{graphicx}
\usepackage{setspace}
\usepackage[version=4]{mhchem}

\newcommand{\dbquote}[1]{\textquotedblleft#1\textquotedblright}
\newcommand{\blacktr}[0]{\item[$\blacktriangleright$]}
\newcommand{\emptyc}[0]{\item[$\circ$]}

\setlength{\grammarindent}{2cm}

\lstset{
  basicstyle=\itshape,
  xleftmargin=3em,
  literate={:=}{$\rightarrow$}{2}
           {α}{$\alpha$}{1}
           {δ}{$\delta$}{1}
}


\linespread{1.3}

\author{
        Dzikiewicz, Luis Enrique\\
        Legajo: D-3850/4\\
        \texttt{luisdzi.87@gmail.com}
        \and Ernandorena, Iván\\
        Legajo: E-1115/1\\
        \texttt{ivan.ernandorena@gmail.com}
        \and Güella, Julio\\
        Legajo: G-5061/1\\
        \texttt{julioguella@hotmail.com }
}

\date{
    Fecha de entrega: -2017
}

\title {
    \Huge  \textsc{Trabajo Práctico Final\\}
    \large \textsc{Sistemas Operativos I}
}

\begin{document}


    \pagenumbering{gobble}

    \maketitle

    \thispagestyle{empty}

    \begin{center}
         \large \bf Docentes
    \end{center}

    \begin{center}
      Guido Macchi
      
      Guillermo Grinblat

      José Luis Díaz
        \vspace{2cm}

        \includegraphics[scale=1.5]{Logo-Unr}
    

    \end{center}


\newpage

%------------------------------------------------------------------------------


\pagenumbering{arabic}

\section*{Diseño del trabajo práctico}

El trabajo práctico consta de dos archivos que permiten su funcionamiento:
\begin{itemize}
  \blacktr \texttt{server.erl}: En él se encuentra todo lo relacionado a los servidores, conexiones, juegos y comandos.
  \blacktr \texttt{game.erl}: Se encarga de la interfaz y de la parte visual relacionada a el juego de TA-TE-TI, por ejemplo, se encuentran las funciones encargadas de imprimir la ayuda, el tablero, chequear cuando un jugador gana, etc. 
\end{itemize}

\subsection*{Conceptos de diseño}
\begin{itemize}
  \blacktr Al iniciar un servidor, se crea una lista donde llevar los clientes llamada \texttt{list_of_client}, que se registra globalmente con el nombre de \texttt{clients_pid} y una para llevar las partidas en curso llamada \texttt{lists_of_games}, registrada como \texttt{games_pid}. Además, se da inicio y registro a los procesos de \texttt{pbalance} y \texttt{pstat}, asi como al \texttt{dispatcher}.
  \blacktr Con respecto al balanceo, la función \texttt{pstat} se encarga de obtener las cargas de los nodos e informarles a los demás nodos sobre las cargas antes obtenidas. Luego, ejecuta \texttt{pbalance} sobre cada nodo y le envia un mensaje con la carga del nodo previamente obtenida con la función \texttt{statistics}. En \texttt{pbalance}, si la carga obtenida por el mensaje es menor a la carga del nodo en la que estoy parado, entonces... Preguntar a Ivan ?)...
  

\end{itemize}

\end{document}