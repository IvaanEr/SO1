
\documentclass[a4paper]{article}
\usepackage[utf8]{inputenc}
\usepackage{amsmath,amssymb,latexsym}
\usepackage{syntax}
\usepackage[margin=1.5cm]{geometry}
\usepackage{amssymb}
\usepackage{tipa}
\usepackage{listings}
\usepackage{graphicx}
\usepackage{setspace}
\usepackage{hyperref}
\usepackage[version=4]{mhchem}

\newcommand{\dbquote}[1]{\textquotedblleft#1\textquotedblright}
\newcommand{\blacktr}[0]{\item[$\blacktriangleright$]}
\newcommand{\emptyc}[0]{\item[$\circ$]}

\setlength{\grammarindent}{2cm}

\lstset{
  basicstyle=\itshape,
  xleftmargin=3em,
  literate={:=}{$\rightarrow$}{2}
           {α}{$\alpha$}{1}
           {δ}{$\delta$}{1}
}


\linespread{1.3}

\author{
        Dzikiewicz, Luis Enrique\\
        Legajo: D-3850/4\\
        \texttt{luisdzi.87@gmail.com}
        \and Ernandorena, Iván\\
        Legajo: E-1115/1\\
        \texttt{ivan.ernandorena@gmail.com}
        \and Güella, Julio\\
        Legajo: G-5061/1\\
        \texttt{julioguella@hotmail.com }
}

\date{
    Fecha de entrega: -2017
}

\title {
    \Huge  \textsc{Trabajo Práctico Final\\}
    \large \textsc{Sistemas Operativos I}
}

\begin{document}


    \pagenumbering{gobble}

    \maketitle

    \thispagestyle{empty}

    \begin{center}
         \large \bf Docentes
    \end{center}

    \begin{center}
      Guido Macchi
      
      Guillermo Grinblat

      José Luis Díaz
        \vspace{2cm}

        \includegraphics[scale=1.5]{Logo-Unr}
    

    \end{center}


\newpage

%------------------------------------------------------------------------------


\pagenumbering{arabic}

\section*{Diseño del trabajo práctico}

El trabajo práctico consta de dos archivos que permiten su funcionamiento:
\begin{itemize}
  \blacktr \texttt{server.erl}: En él se encuentra todo lo relacionado a los servidores. Se tratan las conexiones de los usuarios, creación de juegos, jugadas, y demás comandos.
  \blacktr \texttt{game.erl}: Se encarga de la interfaz y de la parte visual relacionada a el juego de TA-TE-TI, por ejemplo, se encuentran las funciones encargadas de imprimir la ayuda, el tablero, chequear cuando un jugador gana, etc. 
\end{itemize}

\subsection*{Conceptos de diseño}
\begin{itemize}
  \blacktr Al inicializar un servidor, se crea un nuevo proceso donde se llevará una lista con los nombres de los clientes de todo el sistema distribuido llamado \texttt{list_of_client}, que se registra globalmente con el nombre de \texttt{clients_pid} para poder acceder a este proceso desde cualquier nodo. Se realiza lo mismo, para un proceso registrado globalmente como \texttt{games_pid} para llevar los nombres de las partidas en curso. Además tanto los clientes como las partidas en curso han sido registradas globalmente. También, se da inicio y registro local a los procesos de \texttt{pbalance} y \texttt{pstat}. Luego de iniciar estos procesos claves para el almacenamiento de información y funcionamiento del sistema se pasa a la función \texttt{dispatcher} con el Socket donde el nodo recibe conexiones entrantes.
  \blacktr Con respecto al balanceo, la función \texttt{pstat} se encarga de enviarle a la función \texttt{pbalance} de todos los nodos el estado de carga de él mismo. Luego, \texttt{pbalance} recibe el estado de carga de todos los nodos y se queda con el nodo de menor carga, así cuando un proceso desea saber que nodo es el de menor carga, \texttt{pbalance} le envía este nodo.
  \blacktr Un juego consta de sus jugadores, observadores, el tablero de juego, y de quien es el turno actual. Los jugadores dentro de un juego estan almacenados de la forma \texttt{\{N,Jugador\}}, donde \texttt{N} es un 1 o un 2 correspondiente al turno y \texttt{Jugador} es el nombre registrado globalmente (un átomo). Por diseño, el usuario que crea el juego con el comando \texttt{NEW} va a ser el jugador número 1 y el turno se establece aleatoriamente.
  \blacktr Cuando se realiza un cambio en un juego, el servidor envía automáticamente a cada jugador y observador de ese juego el nuevo tablero indicando quien fue el jugador que realizó esa jugada. Se obvió el comando \texttt{UPD} por esto, ya que esta alternativa parece más clara y eficiente para los usuarios. 
\end{itemize}

\subsection*{Inicializar el sistema}
\begin{itemize}
    \blacktr Antes de inicializar la consola de Erlang, ejecutamos el comando \texttt{\textdollar epmd -daemon} por si el comando \texttt{\textdollar erl} arranca automaticamente el epmd (Erlang  Port  Mapper  Daemonepmd). Es para evitar el error "register/listen error: econnrefused".
    \blacktr Antes de abrir la terminal de Erlang con \texttt{\textdollar erl} debemos conocer la IP del equipo en la red. Una de las formas de obtenerla es con el comando \texttt{\textdollar ipconfig}. Una vez tenemos la IP abrimos la terminal de Erlang de la siguiente manera: \texttt{erl -name nodo@IP}, donde \texttt{nodo} es el nombre que le queremos dar al mismo y \texttt{IP} la que se obtuvo anteriormente.
    \blacktr Compilamos el servidor con \texttt{1\textgreater c(server).} El servidor se arranca con la función \texttt{start\textbackslash2} del módulo \texttt{server}. Esta función toma como parametros un nombre para el nodo (puede ser el mismo con el que se abrio el shell), un puerto disponible y una lista con el nombre de los demás nodos de la forma \texttt{nodo@IP} que en principio puede ser vacia ya que no hay otros nodos en el sistema. 

    \blacktr El sistema deberia sincronizarse automaticamente con los demas nodos de la red (si los hubiera), para chequear esto se puede utilizar la funcion \texttt{nodes()} que nos devuelve una lista de todos los demas nodos del sistema.
\end{itemize}

\subsection*{Como jugar}
\begin{itemize}
  \blacktr Una vez que el sistema distribuido, con uno o mas nodos, está corriendo, se puede conectar a este desde otra terminal a traves del protocolo \texttt{telnet}. Para conectarse se debe conocer la IP de uno de los nodos y el puerto donde trabaja. \texttt{telnet IP Puerto} basta para conectarse al sistema.
  \blacktr Lo primero que se debe hacer es registrarse con un nombre de usuario (\texttt{CON 'nombre'}). Luego de esto se puede comenzar viendo los juegos disponibles con \texttt{LSG} o creando un nuevo juego \texttt{NEW 'nombre juego'}. Para ver todos los comandos disponibles una vez conectado está disponible la ayuda con \texttt{HELP}.
  \blacktr Al estar permitido participar en más de un juego a la vez. Cada vez que se muestra un tablero de un juego, en la parte superior se muestra el nombre del juego y quienes son sus jugadores junto con que simbolo tiene asignado cada uno. Además se indica quien fue el último en realizar una jugada, sobre todo para que los observadores puedan seguir mejor el progreso de una partida y si un jugador está jugando varias partidas simultaneamente pueda saber con exactitud a que juego pertenece el tablero que se le muestra.
\end{itemize}

\end{document}
